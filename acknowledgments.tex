I cannot help but follow standard protocol and begin these acknowledgements by expressing my gratitude to my adviser, Dr. John Willinsky. In his duties as an adviser, John advises with equal parts wisdom and wit; he observes and critiques with an eye for both the big picture and for the character spacing; and he does it all without needing to put down the \emph{New York Times}. Through John's friendship, I have gained a deeper appreciation for many of the makings of a great scholar.\footnote{Namely, footnotes, espresso (with no milk), \emph{The New Yorker}, Moe's books, and John Locke.} That said, John has been so much more than an adviser, he has been a mentor, a role model, and a friend. I am grateful to have had him as a guide in all manner of professional and personal things.

I wish I could end there, but I must also acknowledge John's role in taking this "street urchin" off the streets of Buenos Aires in 2007. By hiring me, sight unseen, he set off a chain of events that have culminated in thousands of airmiles, a wife, a son, a career, and most recently, a PhD.\footnote{These are listed chronological order, not in order of importance.} For all of this, I am forever in his debt.

There is, of course, one other person who has forever altered the course of my life: my wife. Maura Patricia Camino Aparicio has kept life interesting by either following me or initiating her own adventures since that fateful day during a Italy-USA match. Her love, support, companionship, and outlandish ideas (lets move to Mexico!) have kept me from becoming (even more of) a curmudgeon.\footnote{I am especially grateful she did not kill me when I forgot to renew our housing contract and forced us to pack up and move within two weeks. Gracias agente.}

I would also like to thank Dr. Gustavo Fischman, whose suggestion in a bar in Toluca planted the idea in my mind that a PhD might be a good idea.\footnote{Before that moment, I had sworn I would never get another degree, and even had a standing bet it would never happen—Lisa, we will need to come up with a payment plan.} Gustavo has always treated me like an equal, even when I clearly was not. I am lucky to have him as a colleague and a friend. Similarly, I need to thank everyone in the McLab, but especially Charlie Gomez, Susan Biancani, Elina Mäkinen, and Eliza Evans. They gave me the academic home (literally and metaphorically) that finally gave me a sense of belonging at the GSE.

I am also forever indebted to everyone in Latin America working to improve the quality and visibility of Latin American scholarship. This dissertation is inspired by you, and is dedicated to you. This includes, but is not limited to, the efforts of RedALyC and SciELO, whose insights, technical knowhow, and cooperation made this work possible. This research was also made possible by funding from the International Development Research Centre (IDRC Grant reference number: 106660-001).

Finally, I would be remiss if I didn't thank Kenneth Shores, whose friendship came just as I was beginning to think the only way to get a beer at Stanford was to send out an evite.\footnote{If a friend's value can be measured in the number of times they shirk responsibility to spend time with you, then Ken is the most valuable friend I have.} Thank you for all the good times (and for all the favors).







