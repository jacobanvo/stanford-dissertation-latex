This study explores the extent to which research published in Latin America—where the vast majority of which is made freely available to the public—has an impact and reach beyond the academic community. It addresses the ways in which the study of research impact is moving beyond the counting citations, which has dominated bibliometrics for well over the last 50 years. As more of the world's research is made freely available to the public, there is an increasing probability that the impact and reach of research extends beyond the confines of academia. To establish the current extent of public access, this study explores who the users of Latin American research are, as well as their motivations for accessing the work by using a series of simple pop-up surveys, which were displayed to users of the two largest scholarly journal portals in Latin America. The results, after thousands of responses, indicate that traditional scholarly use makes up only a quarter of the total use in Latin America. The majority of use is from non-scholar communities, namely students (around 50\% of the total use) and from individuals interested for professional or personal reasons (collectively around 20\% of the total use). By linking the survey responses to the articles being read, it was also possible to identify points of convergence and divergence in student, faculty, and public interest groups. Finally, this study employed methods from a new field of inquiry, altmetrics, in an attempt to capture engagement with research on the social Web. The success of such methods for the Latin American case were limited due to low coverage levels, but the research nevertheless contributes to the understanding of nascent field of altmetrics more broadly. The study concludes with a discussion of the conceptual, political, curricular, and methodological implications of this new approach to scientific communication.